\chapter*{Introduction}

Durante los últimas décadas la simulación computacional se ha convertido en una poderosa herramienta de análisis a través de múltiples ramas de la ingeniería. Esta disciplina comprende tanto el procesamiento de datos de campo, como la creación de modelos computacionales relacionados con fenómenos físicos o problemas ingenieriles con el fin de buscar condiciones óptimas de seguridad, economía y funcionalidad entre otras.  Dichas herramientas son de especial interés en problemas de consultoría donde, por su grado de dificultad, los temas abordados no se encuentran aún suficientemente documentados en textos o normas de análisis y diseño.  El ingeniero civil moderno debe estar entonces capacitado para desarrollar simulaciones de problemas relativos a las diferentes áreas de la ingeniería, especialmente aquellas fundamentadas en una base mecánica.

Este documento constituye las notas de clase para el curso IC0285 Modelación Computacional, en el Departamento de Ingeniería Civil de la Universidad EAFIT. 
El curso busca proporcionar al estudiante de ingeniería civil las competencias necesarias para abordar un problema con base en los principios de la mecánica aplicada mediante simulaciones computacionales. Al terminar el curso el estudiante deberá estar en capacidad de identificar o proponer un modelo matemático para describir el problema; formular un método de solución numérica del problema; desarrollar y codificar el método de solución; validar y verificar el programa resultante y aplicar el programa resultante para determinar la solución de un problema complejo.

No se asume un conocimiento previo en programación. Sin embargo, el estudiante debe estar en la capacidad de: calcular derivadas e integrales para funciones simples, conocer algunas aplicaciones de las derivadas e integrales, saber que algunos problemas de física e ingeniería pueden formularse como sistemas de ecuaciones diferenciales. Además, debe tener unas destrezas mínimas en el uso del computador como: instalar nuevo software en su computador personal y manipular archivos en su sistemas operativo.